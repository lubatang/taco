\begin{abstract}
This paper describes MCLinker, a linker framework designed to support {\em whole-program analysis and optimization} by providing a global system view to transformations. MCLinker defines a {\em generic representation} of system modules, a fragment-reference graph. Fragment-reference graph is characterized by the use-define relationships among all modules. It allows transformations to perform analysis and optimization with universal information on all modules in the program. The address information in the fragment-reference graph enables more aggressive analysis and optimization that traditional compilers cannot employ, such as branch-island elimination and instruction relaxing. In addition to the module representation, MCLinker also defines four {\em distinguishing transformation phases} for sophisticated analysis and optimization: normalization, resolution, layout and emission. To the best of our knowledge, MCLinker is the first linker who separates the linking process into clear phases and defines these phases in mathematic representation. Based on the mathematical definitions, we develop delicate, {\em fast linking algorithms} with small memory usage. The MCLinker generic representation, distinquishing transformation phases and fast algorithms together provide the capacity for whole-program analysis and optimization. We describe the design of MCLinker framework, algorithms and representations and evaluate the design in three ways: (a) linking speed (b) the size and effectiveness of the representation, and (c) the lines of code and number of supported platforms. In our experiments, MCLinker runs steadily 100\% faster than GNU ld, and 30\% faster than the Google gold linker. MCLinker spends less time to handle symbols and relocations with smaller memory usage. MCLinker's lines of code are only 20\% of GNU ld's and 50\% of the Google gold linker's. Furthermore, MCLinker has already supported more target platforms than the Google gold linker.
\end{abstract}
